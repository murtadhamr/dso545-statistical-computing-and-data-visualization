\documentclass[11pt]{article}

    \usepackage[breakable]{tcolorbox}
    \usepackage{parskip} % Stop auto-indenting (to mimic markdown behaviour)
    

    % Basic figure setup, for now with no caption control since it's done
    % automatically by Pandoc (which extracts ![](path) syntax from Markdown).
    \usepackage{graphicx}
    % Maintain compatibility with old templates. Remove in nbconvert 6.0
    \let\Oldincludegraphics\includegraphics
    % Ensure that by default, figures have no caption (until we provide a
    % proper Figure object with a Caption API and a way to capture that
    % in the conversion process - todo).
    \usepackage{caption}
    \DeclareCaptionFormat{nocaption}{}
    \captionsetup{format=nocaption,aboveskip=0pt,belowskip=0pt}

    \usepackage{float}
    \floatplacement{figure}{H} % forces figures to be placed at the correct location
    \usepackage{xcolor} % Allow colors to be defined
    \usepackage{enumerate} % Needed for markdown enumerations to work
    \usepackage{geometry} % Used to adjust the document margins
    \usepackage{amsmath} % Equations
    \usepackage{amssymb} % Equations
    \usepackage{textcomp} % defines textquotesingle
    % Hack from http://tex.stackexchange.com/a/47451/13684:
    \AtBeginDocument{%
        \def\PYZsq{\textquotesingle}% Upright quotes in Pygmentized code
    }
    \usepackage{upquote} % Upright quotes for verbatim code
    \usepackage{eurosym} % defines \euro

    \usepackage{iftex}
    \ifPDFTeX
        \usepackage[T1]{fontenc}
        \IfFileExists{alphabeta.sty}{
              \usepackage{alphabeta}
          }{
              \usepackage[mathletters]{ucs}
              \usepackage[utf8x]{inputenc}
          }
    \else
        \usepackage{fontspec}
        \usepackage{unicode-math}
    \fi

    \usepackage{fancyvrb} % verbatim replacement that allows latex
    \usepackage{grffile} % extends the file name processing of package graphics
                         % to support a larger range
    \makeatletter % fix for old versions of grffile with XeLaTeX
    \@ifpackagelater{grffile}{2019/11/01}
    {
      % Do nothing on new versions
    }
    {
      \def\Gread@@xetex#1{%
        \IfFileExists{"\Gin@base".bb}%
        {\Gread@eps{\Gin@base.bb}}%
        {\Gread@@xetex@aux#1}%
      }
    }
    \makeatother
    \usepackage[Export]{adjustbox} % Used to constrain images to a maximum size
    \adjustboxset{max size={0.9\linewidth}{0.9\paperheight}}

    % The hyperref package gives us a pdf with properly built
    % internal navigation ('pdf bookmarks' for the table of contents,
    % internal cross-reference links, web links for URLs, etc.)
    \usepackage{hyperref}
    % The default LaTeX title has an obnoxious amount of whitespace. By default,
    % titling removes some of it. It also provides customization options.
    \usepackage{titling}
    \usepackage{longtable} % longtable support required by pandoc >1.10
    \usepackage{booktabs}  % table support for pandoc > 1.12.2
    \usepackage{array}     % table support for pandoc >= 2.11.3
    \usepackage{calc}      % table minipage width calculation for pandoc >= 2.11.1
    \usepackage[inline]{enumitem} % IRkernel/repr support (it uses the enumerate* environment)
    \usepackage[normalem]{ulem} % ulem is needed to support strikethroughs (\sout)
                                % normalem makes italics be italics, not underlines
    \usepackage{mathrsfs}
    

    
    % Colors for the hyperref package
    \definecolor{urlcolor}{rgb}{0,.145,.698}
    \definecolor{linkcolor}{rgb}{.71,0.21,0.01}
    \definecolor{citecolor}{rgb}{.12,.54,.11}

    % ANSI colors
    \definecolor{ansi-black}{HTML}{3E424D}
    \definecolor{ansi-black-intense}{HTML}{282C36}
    \definecolor{ansi-red}{HTML}{E75C58}
    \definecolor{ansi-red-intense}{HTML}{B22B31}
    \definecolor{ansi-green}{HTML}{00A250}
    \definecolor{ansi-green-intense}{HTML}{007427}
    \definecolor{ansi-yellow}{HTML}{DDB62B}
    \definecolor{ansi-yellow-intense}{HTML}{B27D12}
    \definecolor{ansi-blue}{HTML}{208FFB}
    \definecolor{ansi-blue-intense}{HTML}{0065CA}
    \definecolor{ansi-magenta}{HTML}{D160C4}
    \definecolor{ansi-magenta-intense}{HTML}{A03196}
    \definecolor{ansi-cyan}{HTML}{60C6C8}
    \definecolor{ansi-cyan-intense}{HTML}{258F8F}
    \definecolor{ansi-white}{HTML}{C5C1B4}
    \definecolor{ansi-white-intense}{HTML}{A1A6B2}
    \definecolor{ansi-default-inverse-fg}{HTML}{FFFFFF}
    \definecolor{ansi-default-inverse-bg}{HTML}{000000}

    % common color for the border for error outputs.
    \definecolor{outerrorbackground}{HTML}{FFDFDF}

    % commands and environments needed by pandoc snippets
    % extracted from the output of `pandoc -s`
    \providecommand{\tightlist}{%
      \setlength{\itemsep}{0pt}\setlength{\parskip}{0pt}}
    \DefineVerbatimEnvironment{Highlighting}{Verbatim}{commandchars=\\\{\}}
    % Add ',fontsize=\small' for more characters per line
    \newenvironment{Shaded}{}{}
    \newcommand{\KeywordTok}[1]{\textcolor[rgb]{0.00,0.44,0.13}{\textbf{{#1}}}}
    \newcommand{\DataTypeTok}[1]{\textcolor[rgb]{0.56,0.13,0.00}{{#1}}}
    \newcommand{\DecValTok}[1]{\textcolor[rgb]{0.25,0.63,0.44}{{#1}}}
    \newcommand{\BaseNTok}[1]{\textcolor[rgb]{0.25,0.63,0.44}{{#1}}}
    \newcommand{\FloatTok}[1]{\textcolor[rgb]{0.25,0.63,0.44}{{#1}}}
    \newcommand{\CharTok}[1]{\textcolor[rgb]{0.25,0.44,0.63}{{#1}}}
    \newcommand{\StringTok}[1]{\textcolor[rgb]{0.25,0.44,0.63}{{#1}}}
    \newcommand{\CommentTok}[1]{\textcolor[rgb]{0.38,0.63,0.69}{\textit{{#1}}}}
    \newcommand{\OtherTok}[1]{\textcolor[rgb]{0.00,0.44,0.13}{{#1}}}
    \newcommand{\AlertTok}[1]{\textcolor[rgb]{1.00,0.00,0.00}{\textbf{{#1}}}}
    \newcommand{\FunctionTok}[1]{\textcolor[rgb]{0.02,0.16,0.49}{{#1}}}
    \newcommand{\RegionMarkerTok}[1]{{#1}}
    \newcommand{\ErrorTok}[1]{\textcolor[rgb]{1.00,0.00,0.00}{\textbf{{#1}}}}
    \newcommand{\NormalTok}[1]{{#1}}

    % Additional commands for more recent versions of Pandoc
    \newcommand{\ConstantTok}[1]{\textcolor[rgb]{0.53,0.00,0.00}{{#1}}}
    \newcommand{\SpecialCharTok}[1]{\textcolor[rgb]{0.25,0.44,0.63}{{#1}}}
    \newcommand{\VerbatimStringTok}[1]{\textcolor[rgb]{0.25,0.44,0.63}{{#1}}}
    \newcommand{\SpecialStringTok}[1]{\textcolor[rgb]{0.73,0.40,0.53}{{#1}}}
    \newcommand{\ImportTok}[1]{{#1}}
    \newcommand{\DocumentationTok}[1]{\textcolor[rgb]{0.73,0.13,0.13}{\textit{{#1}}}}
    \newcommand{\AnnotationTok}[1]{\textcolor[rgb]{0.38,0.63,0.69}{\textbf{\textit{{#1}}}}}
    \newcommand{\CommentVarTok}[1]{\textcolor[rgb]{0.38,0.63,0.69}{\textbf{\textit{{#1}}}}}
    \newcommand{\VariableTok}[1]{\textcolor[rgb]{0.10,0.09,0.49}{{#1}}}
    \newcommand{\ControlFlowTok}[1]{\textcolor[rgb]{0.00,0.44,0.13}{\textbf{{#1}}}}
    \newcommand{\OperatorTok}[1]{\textcolor[rgb]{0.40,0.40,0.40}{{#1}}}
    \newcommand{\BuiltInTok}[1]{{#1}}
    \newcommand{\ExtensionTok}[1]{{#1}}
    \newcommand{\PreprocessorTok}[1]{\textcolor[rgb]{0.74,0.48,0.00}{{#1}}}
    \newcommand{\AttributeTok}[1]{\textcolor[rgb]{0.49,0.56,0.16}{{#1}}}
    \newcommand{\InformationTok}[1]{\textcolor[rgb]{0.38,0.63,0.69}{\textbf{\textit{{#1}}}}}
    \newcommand{\WarningTok}[1]{\textcolor[rgb]{0.38,0.63,0.69}{\textbf{\textit{{#1}}}}}


    % Define a nice break command that doesn't care if a line doesn't already
    % exist.
    \def\br{\hspace*{\fill} \\* }
    % Math Jax compatibility definitions
    \def\gt{>}
    \def\lt{<}
    \let\Oldtex\TeX
    \let\Oldlatex\LaTeX
    \renewcommand{\TeX}{\textrm{\Oldtex}}
    \renewcommand{\LaTeX}{\textrm{\Oldlatex}}
    % Document parameters
    % Document title
    \title{DSO545\_HW5-MuhammadMurtadhaRamadhan}
    
    
    
    
    
% Pygments definitions
\makeatletter
\def\PY@reset{\let\PY@it=\relax \let\PY@bf=\relax%
    \let\PY@ul=\relax \let\PY@tc=\relax%
    \let\PY@bc=\relax \let\PY@ff=\relax}
\def\PY@tok#1{\csname PY@tok@#1\endcsname}
\def\PY@toks#1+{\ifx\relax#1\empty\else%
    \PY@tok{#1}\expandafter\PY@toks\fi}
\def\PY@do#1{\PY@bc{\PY@tc{\PY@ul{%
    \PY@it{\PY@bf{\PY@ff{#1}}}}}}}
\def\PY#1#2{\PY@reset\PY@toks#1+\relax+\PY@do{#2}}

\@namedef{PY@tok@w}{\def\PY@tc##1{\textcolor[rgb]{0.73,0.73,0.73}{##1}}}
\@namedef{PY@tok@c}{\let\PY@it=\textit\def\PY@tc##1{\textcolor[rgb]{0.24,0.48,0.48}{##1}}}
\@namedef{PY@tok@cp}{\def\PY@tc##1{\textcolor[rgb]{0.61,0.40,0.00}{##1}}}
\@namedef{PY@tok@k}{\let\PY@bf=\textbf\def\PY@tc##1{\textcolor[rgb]{0.00,0.50,0.00}{##1}}}
\@namedef{PY@tok@kp}{\def\PY@tc##1{\textcolor[rgb]{0.00,0.50,0.00}{##1}}}
\@namedef{PY@tok@kt}{\def\PY@tc##1{\textcolor[rgb]{0.69,0.00,0.25}{##1}}}
\@namedef{PY@tok@o}{\def\PY@tc##1{\textcolor[rgb]{0.40,0.40,0.40}{##1}}}
\@namedef{PY@tok@ow}{\let\PY@bf=\textbf\def\PY@tc##1{\textcolor[rgb]{0.67,0.13,1.00}{##1}}}
\@namedef{PY@tok@nb}{\def\PY@tc##1{\textcolor[rgb]{0.00,0.50,0.00}{##1}}}
\@namedef{PY@tok@nf}{\def\PY@tc##1{\textcolor[rgb]{0.00,0.00,1.00}{##1}}}
\@namedef{PY@tok@nc}{\let\PY@bf=\textbf\def\PY@tc##1{\textcolor[rgb]{0.00,0.00,1.00}{##1}}}
\@namedef{PY@tok@nn}{\let\PY@bf=\textbf\def\PY@tc##1{\textcolor[rgb]{0.00,0.00,1.00}{##1}}}
\@namedef{PY@tok@ne}{\let\PY@bf=\textbf\def\PY@tc##1{\textcolor[rgb]{0.80,0.25,0.22}{##1}}}
\@namedef{PY@tok@nv}{\def\PY@tc##1{\textcolor[rgb]{0.10,0.09,0.49}{##1}}}
\@namedef{PY@tok@no}{\def\PY@tc##1{\textcolor[rgb]{0.53,0.00,0.00}{##1}}}
\@namedef{PY@tok@nl}{\def\PY@tc##1{\textcolor[rgb]{0.46,0.46,0.00}{##1}}}
\@namedef{PY@tok@ni}{\let\PY@bf=\textbf\def\PY@tc##1{\textcolor[rgb]{0.44,0.44,0.44}{##1}}}
\@namedef{PY@tok@na}{\def\PY@tc##1{\textcolor[rgb]{0.41,0.47,0.13}{##1}}}
\@namedef{PY@tok@nt}{\let\PY@bf=\textbf\def\PY@tc##1{\textcolor[rgb]{0.00,0.50,0.00}{##1}}}
\@namedef{PY@tok@nd}{\def\PY@tc##1{\textcolor[rgb]{0.67,0.13,1.00}{##1}}}
\@namedef{PY@tok@s}{\def\PY@tc##1{\textcolor[rgb]{0.73,0.13,0.13}{##1}}}
\@namedef{PY@tok@sd}{\let\PY@it=\textit\def\PY@tc##1{\textcolor[rgb]{0.73,0.13,0.13}{##1}}}
\@namedef{PY@tok@si}{\let\PY@bf=\textbf\def\PY@tc##1{\textcolor[rgb]{0.64,0.35,0.47}{##1}}}
\@namedef{PY@tok@se}{\let\PY@bf=\textbf\def\PY@tc##1{\textcolor[rgb]{0.67,0.36,0.12}{##1}}}
\@namedef{PY@tok@sr}{\def\PY@tc##1{\textcolor[rgb]{0.64,0.35,0.47}{##1}}}
\@namedef{PY@tok@ss}{\def\PY@tc##1{\textcolor[rgb]{0.10,0.09,0.49}{##1}}}
\@namedef{PY@tok@sx}{\def\PY@tc##1{\textcolor[rgb]{0.00,0.50,0.00}{##1}}}
\@namedef{PY@tok@m}{\def\PY@tc##1{\textcolor[rgb]{0.40,0.40,0.40}{##1}}}
\@namedef{PY@tok@gh}{\let\PY@bf=\textbf\def\PY@tc##1{\textcolor[rgb]{0.00,0.00,0.50}{##1}}}
\@namedef{PY@tok@gu}{\let\PY@bf=\textbf\def\PY@tc##1{\textcolor[rgb]{0.50,0.00,0.50}{##1}}}
\@namedef{PY@tok@gd}{\def\PY@tc##1{\textcolor[rgb]{0.63,0.00,0.00}{##1}}}
\@namedef{PY@tok@gi}{\def\PY@tc##1{\textcolor[rgb]{0.00,0.52,0.00}{##1}}}
\@namedef{PY@tok@gr}{\def\PY@tc##1{\textcolor[rgb]{0.89,0.00,0.00}{##1}}}
\@namedef{PY@tok@ge}{\let\PY@it=\textit}
\@namedef{PY@tok@gs}{\let\PY@bf=\textbf}
\@namedef{PY@tok@gp}{\let\PY@bf=\textbf\def\PY@tc##1{\textcolor[rgb]{0.00,0.00,0.50}{##1}}}
\@namedef{PY@tok@go}{\def\PY@tc##1{\textcolor[rgb]{0.44,0.44,0.44}{##1}}}
\@namedef{PY@tok@gt}{\def\PY@tc##1{\textcolor[rgb]{0.00,0.27,0.87}{##1}}}
\@namedef{PY@tok@err}{\def\PY@bc##1{{\setlength{\fboxsep}{\string -\fboxrule}\fcolorbox[rgb]{1.00,0.00,0.00}{1,1,1}{\strut ##1}}}}
\@namedef{PY@tok@kc}{\let\PY@bf=\textbf\def\PY@tc##1{\textcolor[rgb]{0.00,0.50,0.00}{##1}}}
\@namedef{PY@tok@kd}{\let\PY@bf=\textbf\def\PY@tc##1{\textcolor[rgb]{0.00,0.50,0.00}{##1}}}
\@namedef{PY@tok@kn}{\let\PY@bf=\textbf\def\PY@tc##1{\textcolor[rgb]{0.00,0.50,0.00}{##1}}}
\@namedef{PY@tok@kr}{\let\PY@bf=\textbf\def\PY@tc##1{\textcolor[rgb]{0.00,0.50,0.00}{##1}}}
\@namedef{PY@tok@bp}{\def\PY@tc##1{\textcolor[rgb]{0.00,0.50,0.00}{##1}}}
\@namedef{PY@tok@fm}{\def\PY@tc##1{\textcolor[rgb]{0.00,0.00,1.00}{##1}}}
\@namedef{PY@tok@vc}{\def\PY@tc##1{\textcolor[rgb]{0.10,0.09,0.49}{##1}}}
\@namedef{PY@tok@vg}{\def\PY@tc##1{\textcolor[rgb]{0.10,0.09,0.49}{##1}}}
\@namedef{PY@tok@vi}{\def\PY@tc##1{\textcolor[rgb]{0.10,0.09,0.49}{##1}}}
\@namedef{PY@tok@vm}{\def\PY@tc##1{\textcolor[rgb]{0.10,0.09,0.49}{##1}}}
\@namedef{PY@tok@sa}{\def\PY@tc##1{\textcolor[rgb]{0.73,0.13,0.13}{##1}}}
\@namedef{PY@tok@sb}{\def\PY@tc##1{\textcolor[rgb]{0.73,0.13,0.13}{##1}}}
\@namedef{PY@tok@sc}{\def\PY@tc##1{\textcolor[rgb]{0.73,0.13,0.13}{##1}}}
\@namedef{PY@tok@dl}{\def\PY@tc##1{\textcolor[rgb]{0.73,0.13,0.13}{##1}}}
\@namedef{PY@tok@s2}{\def\PY@tc##1{\textcolor[rgb]{0.73,0.13,0.13}{##1}}}
\@namedef{PY@tok@sh}{\def\PY@tc##1{\textcolor[rgb]{0.73,0.13,0.13}{##1}}}
\@namedef{PY@tok@s1}{\def\PY@tc##1{\textcolor[rgb]{0.73,0.13,0.13}{##1}}}
\@namedef{PY@tok@mb}{\def\PY@tc##1{\textcolor[rgb]{0.40,0.40,0.40}{##1}}}
\@namedef{PY@tok@mf}{\def\PY@tc##1{\textcolor[rgb]{0.40,0.40,0.40}{##1}}}
\@namedef{PY@tok@mh}{\def\PY@tc##1{\textcolor[rgb]{0.40,0.40,0.40}{##1}}}
\@namedef{PY@tok@mi}{\def\PY@tc##1{\textcolor[rgb]{0.40,0.40,0.40}{##1}}}
\@namedef{PY@tok@il}{\def\PY@tc##1{\textcolor[rgb]{0.40,0.40,0.40}{##1}}}
\@namedef{PY@tok@mo}{\def\PY@tc##1{\textcolor[rgb]{0.40,0.40,0.40}{##1}}}
\@namedef{PY@tok@ch}{\let\PY@it=\textit\def\PY@tc##1{\textcolor[rgb]{0.24,0.48,0.48}{##1}}}
\@namedef{PY@tok@cm}{\let\PY@it=\textit\def\PY@tc##1{\textcolor[rgb]{0.24,0.48,0.48}{##1}}}
\@namedef{PY@tok@cpf}{\let\PY@it=\textit\def\PY@tc##1{\textcolor[rgb]{0.24,0.48,0.48}{##1}}}
\@namedef{PY@tok@c1}{\let\PY@it=\textit\def\PY@tc##1{\textcolor[rgb]{0.24,0.48,0.48}{##1}}}
\@namedef{PY@tok@cs}{\let\PY@it=\textit\def\PY@tc##1{\textcolor[rgb]{0.24,0.48,0.48}{##1}}}

\def\PYZbs{\char`\\}
\def\PYZus{\char`\_}
\def\PYZob{\char`\{}
\def\PYZcb{\char`\}}
\def\PYZca{\char`\^}
\def\PYZam{\char`\&}
\def\PYZlt{\char`\<}
\def\PYZgt{\char`\>}
\def\PYZsh{\char`\#}
\def\PYZpc{\char`\%}
\def\PYZdl{\char`\$}
\def\PYZhy{\char`\-}
\def\PYZsq{\char`\'}
\def\PYZdq{\char`\"}
\def\PYZti{\char`\~}
% for compatibility with earlier versions
\def\PYZat{@}
\def\PYZlb{[}
\def\PYZrb{]}
\makeatother


    % For linebreaks inside Verbatim environment from package fancyvrb.
    \makeatletter
        \newbox\Wrappedcontinuationbox
        \newbox\Wrappedvisiblespacebox
        \newcommand*\Wrappedvisiblespace {\textcolor{red}{\textvisiblespace}}
        \newcommand*\Wrappedcontinuationsymbol {\textcolor{red}{\llap{\tiny$\m@th\hookrightarrow$}}}
        \newcommand*\Wrappedcontinuationindent {3ex }
        \newcommand*\Wrappedafterbreak {\kern\Wrappedcontinuationindent\copy\Wrappedcontinuationbox}
        % Take advantage of the already applied Pygments mark-up to insert
        % potential linebreaks for TeX processing.
        %        {, <, #, %, $, ' and ": go to next line.
        %        _, }, ^, &, >, - and ~: stay at end of broken line.
        % Use of \textquotesingle for straight quote.
        \newcommand*\Wrappedbreaksatspecials {%
            \def\PYGZus{\discretionary{\char`\_}{\Wrappedafterbreak}{\char`\_}}%
            \def\PYGZob{\discretionary{}{\Wrappedafterbreak\char`\{}{\char`\{}}%
            \def\PYGZcb{\discretionary{\char`\}}{\Wrappedafterbreak}{\char`\}}}%
            \def\PYGZca{\discretionary{\char`\^}{\Wrappedafterbreak}{\char`\^}}%
            \def\PYGZam{\discretionary{\char`\&}{\Wrappedafterbreak}{\char`\&}}%
            \def\PYGZlt{\discretionary{}{\Wrappedafterbreak\char`\<}{\char`\<}}%
            \def\PYGZgt{\discretionary{\char`\>}{\Wrappedafterbreak}{\char`\>}}%
            \def\PYGZsh{\discretionary{}{\Wrappedafterbreak\char`\#}{\char`\#}}%
            \def\PYGZpc{\discretionary{}{\Wrappedafterbreak\char`\%}{\char`\%}}%
            \def\PYGZdl{\discretionary{}{\Wrappedafterbreak\char`\$}{\char`\$}}%
            \def\PYGZhy{\discretionary{\char`\-}{\Wrappedafterbreak}{\char`\-}}%
            \def\PYGZsq{\discretionary{}{\Wrappedafterbreak\textquotesingle}{\textquotesingle}}%
            \def\PYGZdq{\discretionary{}{\Wrappedafterbreak\char`\"}{\char`\"}}%
            \def\PYGZti{\discretionary{\char`\~}{\Wrappedafterbreak}{\char`\~}}%
        }
        % Some characters . , ; ? ! / are not pygmentized.
        % This macro makes them "active" and they will insert potential linebreaks
        \newcommand*\Wrappedbreaksatpunct {%
            \lccode`\~`\.\lowercase{\def~}{\discretionary{\hbox{\char`\.}}{\Wrappedafterbreak}{\hbox{\char`\.}}}%
            \lccode`\~`\,\lowercase{\def~}{\discretionary{\hbox{\char`\,}}{\Wrappedafterbreak}{\hbox{\char`\,}}}%
            \lccode`\~`\;\lowercase{\def~}{\discretionary{\hbox{\char`\;}}{\Wrappedafterbreak}{\hbox{\char`\;}}}%
            \lccode`\~`\:\lowercase{\def~}{\discretionary{\hbox{\char`\:}}{\Wrappedafterbreak}{\hbox{\char`\:}}}%
            \lccode`\~`\?\lowercase{\def~}{\discretionary{\hbox{\char`\?}}{\Wrappedafterbreak}{\hbox{\char`\?}}}%
            \lccode`\~`\!\lowercase{\def~}{\discretionary{\hbox{\char`\!}}{\Wrappedafterbreak}{\hbox{\char`\!}}}%
            \lccode`\~`\/\lowercase{\def~}{\discretionary{\hbox{\char`\/}}{\Wrappedafterbreak}{\hbox{\char`\/}}}%
            \catcode`\.\active
            \catcode`\,\active
            \catcode`\;\active
            \catcode`\:\active
            \catcode`\?\active
            \catcode`\!\active
            \catcode`\/\active
            \lccode`\~`\~
        }
    \makeatother

    \let\OriginalVerbatim=\Verbatim
    \makeatletter
    \renewcommand{\Verbatim}[1][1]{%
        %\parskip\z@skip
        \sbox\Wrappedcontinuationbox {\Wrappedcontinuationsymbol}%
        \sbox\Wrappedvisiblespacebox {\FV@SetupFont\Wrappedvisiblespace}%
        \def\FancyVerbFormatLine ##1{\hsize\linewidth
            \vtop{\raggedright\hyphenpenalty\z@\exhyphenpenalty\z@
                \doublehyphendemerits\z@\finalhyphendemerits\z@
                \strut ##1\strut}%
        }%
        % If the linebreak is at a space, the latter will be displayed as visible
        % space at end of first line, and a continuation symbol starts next line.
        % Stretch/shrink are however usually zero for typewriter font.
        \def\FV@Space {%
            \nobreak\hskip\z@ plus\fontdimen3\font minus\fontdimen4\font
            \discretionary{\copy\Wrappedvisiblespacebox}{\Wrappedafterbreak}
            {\kern\fontdimen2\font}%
        }%

        % Allow breaks at special characters using \PYG... macros.
        \Wrappedbreaksatspecials
        % Breaks at punctuation characters . , ; ? ! and / need catcode=\active
        \OriginalVerbatim[#1,codes*=\Wrappedbreaksatpunct]%
    }
    \makeatother

    % Exact colors from NB
    \definecolor{incolor}{HTML}{303F9F}
    \definecolor{outcolor}{HTML}{D84315}
    \definecolor{cellborder}{HTML}{CFCFCF}
    \definecolor{cellbackground}{HTML}{F7F7F7}

    % prompt
    \makeatletter
    \newcommand{\boxspacing}{\kern\kvtcb@left@rule\kern\kvtcb@boxsep}
    \makeatother
    \newcommand{\prompt}[4]{
        {\ttfamily\llap{{\color{#2}[#3]:\hspace{3pt}#4}}\vspace{-\baselineskip}}
    }
    

    
    % Prevent overflowing lines due to hard-to-break entities
    \sloppy
    % Setup hyperref package
    \hypersetup{
      breaklinks=true,  % so long urls are correctly broken across lines
      colorlinks=true,
      urlcolor=urlcolor,
      linkcolor=linkcolor,
      citecolor=citecolor,
      }
    % Slightly bigger margins than the latex defaults
    
    \geometry{verbose,tmargin=1in,bmargin=1in,lmargin=1in,rmargin=1in}
    
    

\begin{document}
    
    \maketitle
    
    

    
    \section{Homework 5}\label{homework-5}

\#\#\# Due: Tuesday Nov 21, at 11:59pm via Blackboard

    Import the necessary panda libraries

    \begin{tcolorbox}[breakable, size=fbox, boxrule=1pt, pad at break*=1mm,colback=cellbackground, colframe=cellborder]
\prompt{In}{incolor}{4}{\boxspacing}
\begin{Verbatim}[commandchars=\\\{\}]
\PY{k+kn}{import} \PY{n+nn}{numpy} \PY{k}{as} \PY{n+nn}{np} \PY{c+c1}{\PYZsh{} for mathematical caluclations}
\PY{k+kn}{import} \PY{n+nn}{pandas} \PY{k}{as} \PY{n+nn}{pd} 
\PY{k+kn}{from} \PY{n+nn}{datetime} \PY{k+kn}{import} \PY{n}{datetime}  \PY{c+c1}{\PYZsh{} to access datetime}
\PY{k+kn}{import} \PY{n+nn}{scipy}\PY{n+nn}{.}\PY{n+nn}{stats} \PY{k}{as} \PY{n+nn}{stats}

\PY{c+c1}{\PYZsh{} for data visualization}
\PY{k+kn}{import} \PY{n+nn}{matplotlib}\PY{n+nn}{.}\PY{n+nn}{pyplot} \PY{k}{as} \PY{n+nn}{plt} 
\PY{k+kn}{import} \PY{n+nn}{seaborn} \PY{k}{as} \PY{n+nn}{sns} 
\PY{k+kn}{import} \PY{n+nn}{plotly}\PY{n+nn}{.}\PY{n+nn}{express} \PY{k}{as} \PY{n+nn}{px} \PY{c+c1}{\PYZsh{} for interactive plotting}
\PY{k+kn}{import} \PY{n+nn}{plotly}\PY{n+nn}{.}\PY{n+nn}{graph\PYZus{}objects} \PY{k}{as} \PY{n+nn}{go} \PY{c+c1}{\PYZsh{} for interactive plotting}

\PY{c+c1}{\PYZsh{} set the plot style in matplotlib to ggplot and the firgure size to 15x5\PYZsh{}\PYZsh{} Augmented Dickey Fuller Test for Assessing Stationarity}
\PY{n}{plt}\PY{o}{.}\PY{n}{style}\PY{o}{.}\PY{n}{use}\PY{p}{(}\PY{l+s+s1}{\PYZsq{}}\PY{l+s+s1}{ggplot}\PY{l+s+s1}{\PYZsq{}}\PY{p}{)}
\PY{n}{plt}\PY{o}{.}\PY{n}{rcParams}\PY{p}{[}\PY{l+s+s2}{\PYZdq{}}\PY{l+s+s2}{figure.figsize}\PY{l+s+s2}{\PYZdq{}}\PY{p}{]} \PY{o}{=} \PY{p}{(}\PY{l+m+mi}{15}\PY{p}{,}\PY{l+m+mi}{5}\PY{p}{)}

\PY{c+c1}{\PYZsh{} for ingnoring warnings}
\PY{k+kn}{import} \PY{n+nn}{warnings} \PY{c+c1}{\PYZsh{} to ignore warning}
\PY{n}{warnings}\PY{o}{.}\PY{n}{filterwarnings}\PY{p}{(}\PY{l+s+s1}{\PYZsq{}}\PY{l+s+s1}{ignore}\PY{l+s+s1}{\PYZsq{}}\PY{p}{)}
\end{Verbatim}
\end{tcolorbox}

    Q1. The stacked bar graph below shows the results of Pew Research
Center's study on Trust in different levels of Government by the
American public. Using plotly graph objects, re-create the bar graph
below. (3 points)

    \begin{figure}
\centering
\includegraphics{attachment:newplot\%281\%29.png}
\caption{newplot\%281\%29.png}
\end{figure}

    

    

    \begin{tcolorbox}[breakable, size=fbox, boxrule=1pt, pad at break*=1mm,colback=cellbackground, colframe=cellborder]
\prompt{In}{incolor}{126}{\boxspacing}
\begin{Verbatim}[commandchars=\\\{\}]
\PY{k+kn}{import} \PY{n+nn}{plotly}\PY{n+nn}{.}\PY{n+nn}{graph\PYZus{}objects} \PY{k}{as} \PY{n+nn}{go}

\PY{n}{top\PYZus{}labels} \PY{o}{=} \PY{p}{[}\PY{l+s+s1}{\PYZsq{}}\PY{l+s+s1}{None of\PYZlt{}br\PYZgt{}the Time}\PY{l+s+s1}{\PYZsq{}}\PY{p}{,} \PY{l+s+s1}{\PYZsq{}}\PY{l+s+s1}{Only Some\PYZlt{}br\PYZgt{}of the Time}\PY{l+s+s1}{\PYZsq{}}\PY{p}{,} \PY{l+s+s1}{\PYZsq{}}\PY{l+s+s1}{Just About\PYZlt{}br\PYZgt{}Always}\PY{l+s+s1}{\PYZsq{}}\PY{p}{,} \PY{l+s+s1}{\PYZsq{}}\PY{l+s+s1}{Most of\PYZlt{}br\PYZgt{}the Time}\PY{l+s+s1}{\PYZsq{}}\PY{p}{]}

\PY{c+c1}{\PYZsh{} colors are arranged from darkets to lightest}
\PY{n}{colors} \PY{o}{=} \PY{p}{[}\PY{l+s+s1}{\PYZsq{}}\PY{l+s+s1}{rgba(38, 24, 74, 0.8)}\PY{l+s+s1}{\PYZsq{}}\PY{p}{,} \PY{l+s+s1}{\PYZsq{}}\PY{l+s+s1}{rgba(71, 58, 131, 0.8)}\PY{l+s+s1}{\PYZsq{}}\PY{p}{,}
          \PY{l+s+s1}{\PYZsq{}}\PY{l+s+s1}{rgba(122, 120, 168, 0.8)}\PY{l+s+s1}{\PYZsq{}}\PY{p}{,} \PY{l+s+s1}{\PYZsq{}}\PY{l+s+s1}{rgba(164, 163, 204, 0.85)}\PY{l+s+s1}{\PYZsq{}}\PY{p}{,}
          \PY{l+s+s1}{\PYZsq{}}\PY{l+s+s1}{rgba(190, 192, 213, 1)}\PY{l+s+s1}{\PYZsq{}}\PY{p}{]}

\PY{c+c1}{\PYZsh{} each list have the distribution of responses }
\PY{n}{x\PYZus{}data} \PY{o}{=} \PY{p}{[}\PY{p}{[}\PY{l+m+mi}{4}\PY{p}{,} \PY{l+m+mi}{36}\PY{p}{,} \PY{l+m+mi}{46}\PY{p}{,} \PY{l+m+mi}{12}\PY{p}{]}\PY{p}{,}
          \PY{p}{[}\PY{l+m+mi}{3}\PY{p}{,} \PY{l+m+mi}{29}\PY{p}{,} \PY{l+m+mi}{53}\PY{p}{,} \PY{l+m+mi}{13}\PY{p}{]}\PY{p}{,}
          \PY{p}{[}\PY{l+m+mi}{3}\PY{p}{,} \PY{l+m+mi}{20}\PY{p}{,} \PY{l+m+mi}{56}\PY{p}{,} \PY{l+m+mi}{19}\PY{p}{]}\PY{p}{]}

\PY{c+c1}{\PYZsh{} the list of questions}
\PY{n}{y\PYZus{}data} \PY{o}{=} \PY{p}{[}\PY{l+s+s1}{\PYZsq{}}\PY{l+s+s1}{Local Government}\PY{l+s+s1}{\PYZsq{}}\PY{p}{,}
          \PY{l+s+s1}{\PYZsq{}}\PY{l+s+s1}{State Government}\PY{l+s+s1}{\PYZsq{}}\PY{p}{,} 
          \PY{l+s+s1}{\PYZsq{}}\PY{l+s+s1}{Federal Government}\PY{l+s+s1}{\PYZsq{}}\PY{p}{]}


\PY{n}{fig} \PY{o}{=} \PY{n}{go}\PY{o}{.}\PY{n}{Figure}\PY{p}{(}\PY{p}{)}

\PY{c+c1}{\PYZsh{} outer loop: 5 times (scale: Strongly agree, Agree, etc.)}
\PY{c+c1}{\PYZsh{} inner loop: 4 times (4 questions each with a list of responses)}
\PY{c+c1}{\PYZsh{} for each of the 5 scales (i), and for each question (yd) with its list of responses (xd[i]):}
\PY{c+c1}{\PYZsh{}   \PYZsh{} Create a barchart for each question and its corresponding list of 5 ratings}


\PY{k}{for} \PY{n}{i} \PY{o+ow}{in} \PY{n+nb}{range}\PY{p}{(}\PY{l+m+mi}{0}\PY{p}{,} \PY{n+nb}{len}\PY{p}{(}\PY{n}{x\PYZus{}data}\PY{p}{[}\PY{l+m+mi}{0}\PY{p}{]}\PY{p}{)}\PY{p}{)}\PY{p}{:}
    \PY{k}{for} \PY{n}{xd}\PY{p}{,} \PY{n}{yd} \PY{o+ow}{in} \PY{n+nb}{zip}\PY{p}{(}\PY{n}{x\PYZus{}data}\PY{p}{,} \PY{n}{y\PYZus{}data}\PY{p}{)}\PY{p}{:}
        \PY{n}{fig}\PY{o}{.}\PY{n}{add\PYZus{}trace}\PY{p}{(}\PY{n}{go}\PY{o}{.}\PY{n}{Bar}\PY{p}{(}
            \PY{n}{x}\PY{o}{=}\PY{p}{[}\PY{n}{xd}\PY{p}{[}\PY{n}{i}\PY{p}{]}\PY{p}{]}\PY{p}{,} \PY{n}{y}\PY{o}{=}\PY{p}{[}\PY{n}{yd}\PY{p}{]}\PY{p}{,}
            \PY{n}{orientation}\PY{o}{=}\PY{l+s+s1}{\PYZsq{}}\PY{l+s+s1}{h}\PY{l+s+s1}{\PYZsq{}}\PY{p}{,}
            \PY{n}{marker}\PY{o}{=}\PY{n+nb}{dict}\PY{p}{(}
                \PY{n}{color}\PY{o}{=}\PY{n}{colors}\PY{p}{[}\PY{n}{i}\PY{p}{]}\PY{p}{,}
                \PY{n}{line}\PY{o}{=}\PY{n+nb}{dict}\PY{p}{(}\PY{n}{color}\PY{o}{=}\PY{l+s+s1}{\PYZsq{}}\PY{l+s+s1}{rgb(248, 248, 249)}\PY{l+s+s1}{\PYZsq{}}\PY{p}{,} \PY{n}{width}\PY{o}{=}\PY{l+m+mi}{1}\PY{p}{)}
            \PY{p}{)}
        \PY{p}{)}\PY{p}{)}

\PY{c+c1}{\PYZsh{} update the layout of the figure: }
\PY{c+c1}{\PYZsh{} 1. get rid of all grids, lines, ticklabels on both x and y axis }
\PY{c+c1}{\PYZsh{} 2. update the bar chart to a stack barchart using `barmode = \PYZsq{}stack\PYZsq{}`}
\PY{c+c1}{\PYZsh{} 3. get rid of the legend }
\PY{c+c1}{\PYZsh{} 4. update the margines lengths}
\PY{c+c1}{\PYZsh{} 5. set the plot and paper background color to white}

\PY{n}{fig}\PY{o}{.}\PY{n}{update\PYZus{}layout}\PY{p}{(}
    \PY{n}{xaxis}\PY{o}{=}\PY{n+nb}{dict}\PY{p}{(}
        \PY{n}{showgrid}\PY{o}{=}\PY{k+kc}{False}\PY{p}{,}
        \PY{n}{showline}\PY{o}{=}\PY{k+kc}{False}\PY{p}{,}
        \PY{n}{showticklabels}\PY{o}{=}\PY{k+kc}{False}\PY{p}{,}
        \PY{n}{zeroline}\PY{o}{=}\PY{k+kc}{False}\PY{p}{,}
        \PY{n}{domain}\PY{o}{=}\PY{p}{[}\PY{l+m+mf}{0.15}\PY{p}{,} \PY{l+m+mi}{1}\PY{p}{]}
    \PY{p}{)}\PY{p}{,}
    \PY{n}{yaxis}\PY{o}{=}\PY{n+nb}{dict}\PY{p}{(}
        \PY{n}{showgrid}\PY{o}{=}\PY{k+kc}{False}\PY{p}{,}
        \PY{n}{showline}\PY{o}{=}\PY{k+kc}{False}\PY{p}{,}
        \PY{n}{showticklabels}\PY{o}{=}\PY{k+kc}{False}\PY{p}{,}
        \PY{n}{zeroline}\PY{o}{=}\PY{k+kc}{False}\PY{p}{,}
    \PY{p}{)}\PY{p}{,}
    \PY{n}{barmode}\PY{o}{=}\PY{l+s+s1}{\PYZsq{}}\PY{l+s+s1}{stack}\PY{l+s+s1}{\PYZsq{}}\PY{p}{,} \PY{c+c1}{\PYZsh{} change type of barchart to stacked}
    \PY{n}{paper\PYZus{}bgcolor}\PY{o}{=}\PY{l+s+s1}{\PYZsq{}}\PY{l+s+s1}{rgb(248, 248, 255)}\PY{l+s+s1}{\PYZsq{}}\PY{p}{,}
    \PY{n}{plot\PYZus{}bgcolor}\PY{o}{=}\PY{l+s+s1}{\PYZsq{}}\PY{l+s+s1}{rgb(248, 248, 255)}\PY{l+s+s1}{\PYZsq{}}\PY{p}{,}
    \PY{n}{margin}\PY{o}{=}\PY{n+nb}{dict}\PY{p}{(}\PY{n}{l}\PY{o}{=}\PY{l+m+mi}{120}\PY{p}{,} \PY{n}{r}\PY{o}{=}\PY{l+m+mi}{10}\PY{p}{,} \PY{n}{t}\PY{o}{=}\PY{l+m+mi}{140}\PY{p}{,} \PY{n}{b}\PY{o}{=}\PY{l+m+mi}{80}\PY{p}{)}\PY{p}{,}
    \PY{n}{showlegend}\PY{o}{=}\PY{k+kc}{False}\PY{p}{,}
\PY{p}{)}

\PY{n}{fig}\PY{o}{.}\PY{n}{update\PYZus{}layout}\PY{p}{(}
    \PY{n}{title}\PY{o}{=}\PY{l+s+s2}{\PYZdq{}}\PY{l+s+s2}{\PYZlt{}b\PYZgt{}Trust in Levels of Government\PYZlt{}/b\PYZgt{}}\PY{l+s+s2}{\PYZdq{}}\PY{p}{,}
    \PY{n}{font}\PY{o}{=}\PY{n+nb}{dict}\PY{p}{(}
        \PY{n}{size}\PY{o}{=}\PY{l+m+mi}{20}
    \PY{p}{)}
    
    
\PY{p}{)}

\PY{c+c1}{\PYZsh{} fig.update\PYZus{}layout(xaxis\PYZus{}title=\PYZdq{}X Axis Title\PYZdq{}, font=dict(size=18), xaxis=dict(anchor=\PYZdq{}y2\PYZdq{}))}



\PY{c+c1}{\PYZsh{} let\PYZsq{}s add the annotations to each \PYZsq{}box\PYZsq{} on the graph}
\PY{c+c1}{\PYZsh{} add labels for the axes}

\PY{n}{annotations} \PY{o}{=} \PY{p}{[}\PY{p}{]}
\PY{k}{for} \PY{n}{yd}\PY{p}{,} \PY{n}{xd} \PY{o+ow}{in} \PY{n+nb}{zip}\PY{p}{(}\PY{n}{y\PYZus{}data}\PY{p}{,} \PY{n}{x\PYZus{}data}\PY{p}{)}\PY{p}{:}
    \PY{c+c1}{\PYZsh{} labeling the y\PYZhy{}axis: questions}
    \PY{n}{annotations}\PY{o}{.}\PY{n}{append}\PY{p}{(}\PY{n+nb}{dict}\PY{p}{(}\PY{n}{xref}\PY{o}{=}\PY{l+s+s1}{\PYZsq{}}\PY{l+s+s1}{paper}\PY{l+s+s1}{\PYZsq{}}\PY{p}{,} \PY{c+c1}{\PYZsh{}x = 0.14 in referance to the whole figure (paper)}
                            \PY{n}{yref}\PY{o}{=}\PY{l+s+s1}{\PYZsq{}}\PY{l+s+s1}{y}\PY{l+s+s1}{\PYZsq{}}\PY{p}{,} \PY{c+c1}{\PYZsh{} y= yd in reference to the regular y axis on the plot}
                            \PY{n}{x}\PY{o}{=}\PY{l+m+mf}{0.14}\PY{p}{,} \PY{n}{y}\PY{o}{=}\PY{n}{yd}\PY{p}{,}
                            \PY{n}{xanchor}\PY{o}{=}\PY{l+s+s1}{\PYZsq{}}\PY{l+s+s1}{right}\PY{l+s+s1}{\PYZsq{}}\PY{p}{,}
                            \PY{n}{text}\PY{o}{=}\PY{n+nb}{str}\PY{p}{(}\PY{n}{yd}\PY{p}{)}\PY{p}{,}
                            \PY{n}{font}\PY{o}{=}\PY{n+nb}{dict}\PY{p}{(}\PY{n}{family}\PY{o}{=}\PY{l+s+s1}{\PYZsq{}}\PY{l+s+s1}{Arial}\PY{l+s+s1}{\PYZsq{}}\PY{p}{,} \PY{n}{size}\PY{o}{=}\PY{l+m+mi}{14}\PY{p}{,}
                                      \PY{n}{color}\PY{o}{=}\PY{l+s+s1}{\PYZsq{}}\PY{l+s+s1}{rgb(67, 67, 67)}\PY{l+s+s1}{\PYZsq{}}\PY{p}{)}\PY{p}{,}
                            \PY{n}{showarrow}\PY{o}{=}\PY{k+kc}{False}\PY{p}{,} \PY{n}{align}\PY{o}{=}\PY{l+s+s1}{\PYZsq{}}\PY{l+s+s1}{right}\PY{l+s+s1}{\PYZsq{}}\PY{p}{)}\PY{p}{)}
    
    
    \PY{c+c1}{\PYZsh{} \PYZsh{} labeling the first percentage of each bar (x\PYZus{}axis)}
    \PY{n}{annotations}\PY{o}{.}\PY{n}{append}\PY{p}{(}\PY{n+nb}{dict}\PY{p}{(}\PY{n}{xref}\PY{o}{=}\PY{l+s+s1}{\PYZsq{}}\PY{l+s+s1}{x}\PY{l+s+s1}{\PYZsq{}}\PY{p}{,} \PY{n}{yref}\PY{o}{=}\PY{l+s+s1}{\PYZsq{}}\PY{l+s+s1}{y}\PY{l+s+s1}{\PYZsq{}}\PY{p}{,}
                            \PY{n}{x}\PY{o}{=}\PY{n}{xd}\PY{p}{[}\PY{l+m+mi}{0}\PY{p}{]} \PY{o}{/} \PY{l+m+mi}{2}\PY{p}{,} \PY{n}{y}\PY{o}{=}\PY{n}{yd}\PY{p}{,} \PY{c+c1}{\PYZsh{} to center the position for x, divide by 2}
                            \PY{n}{text}\PY{o}{=}\PY{n+nb}{str}\PY{p}{(}\PY{n}{xd}\PY{p}{[}\PY{l+m+mi}{0}\PY{p}{]}\PY{p}{)} \PY{o}{+} \PY{l+s+s1}{\PYZsq{}}\PY{l+s+s1}{\PYZpc{}}\PY{l+s+s1}{\PYZsq{}}\PY{p}{,} \PY{c+c1}{\PYZsh{} this is the real value of x (the one to show)}
                            \PY{n}{font}\PY{o}{=}\PY{n+nb}{dict}\PY{p}{(}\PY{n}{family}\PY{o}{=}\PY{l+s+s1}{\PYZsq{}}\PY{l+s+s1}{Arial}\PY{l+s+s1}{\PYZsq{}}\PY{p}{,} \PY{n}{size}\PY{o}{=}\PY{l+m+mi}{14}\PY{p}{,}
                                      \PY{n}{color}\PY{o}{=}\PY{l+s+s1}{\PYZsq{}}\PY{l+s+s1}{rgb(248, 248, 255)}\PY{l+s+s1}{\PYZsq{}}\PY{p}{)}\PY{p}{,}
                            \PY{n}{showarrow}\PY{o}{=}\PY{k+kc}{False}\PY{p}{)}\PY{p}{)}

    \PY{c+c1}{\PYZsh{}labeling the first Likert scale (on the top)}
    \PY{k}{if} \PY{n}{yd} \PY{o}{==} \PY{n}{y\PYZus{}data}\PY{p}{[}\PY{o}{\PYZhy{}}\PY{l+m+mi}{1}\PY{p}{]}\PY{p}{:}
        \PY{n}{annotations}\PY{o}{.}\PY{n}{append}\PY{p}{(}\PY{n+nb}{dict}\PY{p}{(}\PY{n}{xref}\PY{o}{=}\PY{l+s+s1}{\PYZsq{}}\PY{l+s+s1}{x}\PY{l+s+s1}{\PYZsq{}}\PY{p}{,} \PY{n}{yref}\PY{o}{=}\PY{l+s+s1}{\PYZsq{}}\PY{l+s+s1}{paper}\PY{l+s+s1}{\PYZsq{}}\PY{p}{,}
                                \PY{n}{x}\PY{o}{=}\PY{n}{xd}\PY{p}{[}\PY{l+m+mi}{0}\PY{p}{]} \PY{o}{/} \PY{l+m+mi}{2}\PY{p}{,} \PY{n}{y}\PY{o}{=}\PY{l+m+mf}{1.3}\PY{p}{,}
                                \PY{n}{text}\PY{o}{=}\PY{n}{top\PYZus{}labels}\PY{p}{[}\PY{l+m+mi}{0}\PY{p}{]}\PY{p}{,}
                                \PY{n}{font}\PY{o}{=}\PY{n+nb}{dict}\PY{p}{(}\PY{n}{family}\PY{o}{=}\PY{l+s+s1}{\PYZsq{}}\PY{l+s+s1}{Arial}\PY{l+s+s1}{\PYZsq{}}\PY{p}{,} \PY{n}{size}\PY{o}{=}\PY{l+m+mi}{14}\PY{p}{,}
                                            \PY{n}{color}\PY{o}{=}\PY{l+s+s1}{\PYZsq{}}\PY{l+s+s1}{rgb(67, 67, 67)}\PY{l+s+s1}{\PYZsq{}}\PY{p}{)}\PY{p}{,}
                                \PY{n}{showarrow}\PY{o}{=}\PY{k+kc}{False}\PY{p}{)}\PY{p}{)}
    
    \PY{c+c1}{\PYZsh{} In the following loop, we go over the rest of labels}
    \PY{n}{space} \PY{o}{=} \PY{n}{xd}\PY{p}{[}\PY{l+m+mi}{0}\PY{p}{]} \PY{c+c1}{\PYZsh{} how much space is needed between each label (Strongly agree, agree, etc.)}
    \PY{k}{for} \PY{n}{i} \PY{o+ow}{in} \PY{n+nb}{range}\PY{p}{(}\PY{l+m+mi}{1}\PY{p}{,} \PY{n+nb}{len}\PY{p}{(}\PY{n}{xd}\PY{p}{)}\PY{p}{)}\PY{p}{:}
            \PY{c+c1}{\PYZsh{} labeling the rest of percentages for each bar (x\PYZus{}axis)}
            \PY{n}{annotations}\PY{o}{.}\PY{n}{append}\PY{p}{(}\PY{n+nb}{dict}\PY{p}{(}\PY{n}{xref}\PY{o}{=}\PY{l+s+s1}{\PYZsq{}}\PY{l+s+s1}{x}\PY{l+s+s1}{\PYZsq{}}\PY{p}{,} \PY{n}{yref}\PY{o}{=}\PY{l+s+s1}{\PYZsq{}}\PY{l+s+s1}{y}\PY{l+s+s1}{\PYZsq{}}\PY{p}{,}
                                    \PY{n}{x}\PY{o}{=}\PY{n}{space} \PY{o}{+} \PY{p}{(}\PY{n}{xd}\PY{p}{[}\PY{n}{i}\PY{p}{]}\PY{o}{/}\PY{l+m+mi}{2}\PY{p}{)}\PY{p}{,} \PY{n}{y}\PY{o}{=}\PY{n}{yd}\PY{p}{,}
                                    \PY{n}{text}\PY{o}{=}\PY{n+nb}{str}\PY{p}{(}\PY{n}{xd}\PY{p}{[}\PY{n}{i}\PY{p}{]}\PY{p}{)} \PY{o}{+} \PY{l+s+s1}{\PYZsq{}}\PY{l+s+s1}{\PYZpc{}}\PY{l+s+s1}{\PYZsq{}}\PY{p}{,}
                                    \PY{n}{font}\PY{o}{=}\PY{n+nb}{dict}\PY{p}{(}\PY{n}{family}\PY{o}{=}\PY{l+s+s1}{\PYZsq{}}\PY{l+s+s1}{Arial}\PY{l+s+s1}{\PYZsq{}}\PY{p}{,} \PY{n}{size}\PY{o}{=}\PY{l+m+mi}{14}\PY{p}{,}
                                                \PY{n}{color}\PY{o}{=}\PY{l+s+s1}{\PYZsq{}}\PY{l+s+s1}{rgb(248, 248, 255)}\PY{l+s+s1}{\PYZsq{}}\PY{p}{)}\PY{p}{,}
                                    \PY{n}{showarrow}\PY{o}{=}\PY{k+kc}{False}\PY{p}{)}\PY{p}{)}
            \PY{c+c1}{\PYZsh{} labeling the Likert scale}
            \PY{k}{if} \PY{n}{yd} \PY{o}{==} \PY{n}{y\PYZus{}data}\PY{p}{[}\PY{o}{\PYZhy{}}\PY{l+m+mi}{1}\PY{p}{]}\PY{p}{:}
                \PY{n}{annotations}\PY{o}{.}\PY{n}{append}\PY{p}{(}\PY{n+nb}{dict}\PY{p}{(}\PY{n}{xref}\PY{o}{=}\PY{l+s+s1}{\PYZsq{}}\PY{l+s+s1}{x}\PY{l+s+s1}{\PYZsq{}}\PY{p}{,} \PY{n}{yref}\PY{o}{=}\PY{l+s+s1}{\PYZsq{}}\PY{l+s+s1}{paper}\PY{l+s+s1}{\PYZsq{}}\PY{p}{,}
                                        \PY{n}{x}\PY{o}{=}\PY{n}{space} \PY{o}{+} \PY{p}{(}\PY{n}{xd}\PY{p}{[}\PY{n}{i}\PY{p}{]}\PY{o}{/}\PY{l+m+mi}{2}\PY{p}{)}\PY{p}{,} \PY{n}{y}\PY{o}{=}\PY{l+m+mf}{1.3}\PY{p}{,}
                                        \PY{n}{text}\PY{o}{=}\PY{n}{top\PYZus{}labels}\PY{p}{[}\PY{n}{i}\PY{p}{]}\PY{p}{,}
                                        \PY{n}{font}\PY{o}{=}\PY{n+nb}{dict}\PY{p}{(}\PY{n}{family}\PY{o}{=}\PY{l+s+s1}{\PYZsq{}}\PY{l+s+s1}{Arial}\PY{l+s+s1}{\PYZsq{}}\PY{p}{,} \PY{n}{size}\PY{o}{=}\PY{l+m+mi}{14}\PY{p}{,}
                                                    \PY{n}{color}\PY{o}{=}\PY{l+s+s1}{\PYZsq{}}\PY{l+s+s1}{rgb(67, 67, 67)}\PY{l+s+s1}{\PYZsq{}}\PY{p}{)}\PY{p}{,}
                                        \PY{n}{showarrow}\PY{o}{=}\PY{k+kc}{False}\PY{p}{)}\PY{p}{)}
            \PY{n}{space} \PY{o}{+}\PY{o}{=} \PY{n}{xd}\PY{p}{[}\PY{n}{i}\PY{p}{]}

\PY{n}{fig}\PY{o}{.}\PY{n}{update\PYZus{}layout}\PY{p}{(}\PY{n}{annotations}\PY{o}{=}\PY{n}{annotations}\PY{p}{)}

\PY{n}{fig}\PY{o}{.}\PY{n}{add\PYZus{}annotation}\PY{p}{(}\PY{n+nb}{dict}\PY{p}{(}\PY{n}{font}\PY{o}{=}\PY{n+nb}{dict}\PY{p}{(}\PY{n}{color}\PY{o}{=}\PY{l+s+s1}{\PYZsq{}}\PY{l+s+s1}{gray}\PY{l+s+s1}{\PYZsq{}}\PY{p}{,}\PY{n}{size}\PY{o}{=}\PY{l+m+mi}{12}\PY{p}{)}\PY{p}{,}
                                        \PY{n}{y}\PY{o}{=}\PY{o}{\PYZhy{}}\PY{l+m+mf}{0.5}\PY{p}{,}
                                        \PY{n}{showarrow}\PY{o}{=}\PY{k+kc}{False}\PY{p}{,}
                                        \PY{n}{text}\PY{o}{=}\PY{l+s+s2}{\PYZdq{}}\PY{l+s+s2}{Source: Pew Research Center}\PY{l+s+s2}{\PYZdq{}}\PY{p}{,}
                                        \PY{n}{textangle}\PY{o}{=}\PY{l+m+mi}{0}\PY{p}{,}
                                        \PY{n}{xanchor}\PY{o}{=}\PY{l+s+s1}{\PYZsq{}}\PY{l+s+s1}{center}\PY{l+s+s1}{\PYZsq{}}\PY{p}{,}
                                        \PY{n}{xref}\PY{o}{=}\PY{l+s+s2}{\PYZdq{}}\PY{l+s+s2}{paper}\PY{l+s+s2}{\PYZdq{}}\PY{p}{,}
                                        \PY{n}{yref}\PY{o}{=}\PY{l+s+s2}{\PYZdq{}}\PY{l+s+s2}{paper}\PY{l+s+s2}{\PYZdq{}}\PY{p}{)}\PY{p}{)}


\PY{n}{fig}\PY{o}{.}\PY{n}{show}\PY{p}{(}\PY{p}{)}
\end{Verbatim}
\end{tcolorbox}

    \begin{center}
    \adjustimage{max size={0.9\linewidth}{0.9\paperheight}}{output_7_0.png}
    \end{center}
    { \hspace*{\fill} \\}
    
    Q2.The plot below shows the total population by county of the states of
Illinois, Michigan and Wisconsin. Import the csv file `population' and
using plotly's choropleth mapbox function, re-create the plot below. (3
points)

    \begin{figure}
\centering
\includegraphics{attachment:newplot\%283\%29.png}
\caption{newplot\%283\%29.png}
\end{figure}

    \begin{tcolorbox}[breakable, size=fbox, boxrule=1pt, pad at break*=1mm,colback=cellbackground, colframe=cellborder]
\prompt{In}{incolor}{120}{\boxspacing}
\begin{Verbatim}[commandchars=\\\{\}]
\PY{k+kn}{from} \PY{n+nn}{urllib}\PY{n+nn}{.}\PY{n+nn}{request} \PY{k+kn}{import} \PY{n}{urlopen}
\PY{k+kn}{import} \PY{n+nn}{json}
\PY{k}{with} \PY{n}{urlopen}\PY{p}{(}\PY{l+s+s1}{\PYZsq{}}\PY{l+s+s1}{https://raw.githubusercontent.com/plotly/datasets/master/geojson\PYZhy{}counties\PYZhy{}fips.json}\PY{l+s+s1}{\PYZsq{}}\PY{p}{)} \PY{k}{as} \PY{n}{response}\PY{p}{:}
    \PY{n}{counties} \PY{o}{=} \PY{n}{json}\PY{o}{.}\PY{n}{load}\PY{p}{(}\PY{n}{response}\PY{p}{)}

\PY{k+kn}{import} \PY{n+nn}{pandas} \PY{k}{as} \PY{n+nn}{pd}
\PY{n}{df} \PY{o}{=} \PY{n}{pd}\PY{o}{.}\PY{n}{read\PYZus{}csv}\PY{p}{(}\PY{l+s+s2}{\PYZdq{}}\PY{l+s+s2}{population.csv}\PY{l+s+s2}{\PYZdq{}}\PY{p}{,} \PY{n}{dtype}\PY{o}{=}\PY{p}{\PYZob{}}\PY{l+s+s2}{\PYZdq{}}\PY{l+s+s2}{FIPS}\PY{l+s+s2}{\PYZdq{}}\PY{p}{:} \PY{n+nb}{str}\PY{p}{\PYZcb{}}\PY{p}{)}  \PY{c+c1}{\PYZsh{} Updated column name to match the CSV file}

\PY{n}{states\PYZus{}to\PYZus{}include} \PY{o}{=} \PY{p}{[}\PY{l+s+s1}{\PYZsq{}}\PY{l+s+s1}{Illinois}\PY{l+s+s1}{\PYZsq{}}\PY{p}{,} \PY{l+s+s1}{\PYZsq{}}\PY{l+s+s1}{Michigan}\PY{l+s+s1}{\PYZsq{}}\PY{p}{,} \PY{l+s+s1}{\PYZsq{}}\PY{l+s+s1}{Wisconsin}\PY{l+s+s1}{\PYZsq{}}\PY{p}{]}  \PY{c+c1}{\PYZsh{} Full state names}

\PY{n}{df\PYZus{}filtered} \PY{o}{=} \PY{n}{df}\PY{p}{[}\PY{n}{df}\PY{p}{[}\PY{l+s+s1}{\PYZsq{}}\PY{l+s+s1}{STNAME}\PY{l+s+s1}{\PYZsq{}}\PY{p}{]}\PY{o}{.}\PY{n}{isin}\PY{p}{(}\PY{n}{states\PYZus{}to\PYZus{}include}\PY{p}{)}\PY{p}{]}

\PY{k+kn}{import} \PY{n+nn}{plotly}\PY{n+nn}{.}\PY{n+nn}{express} \PY{k}{as} \PY{n+nn}{px}

\PY{n}{fig} \PY{o}{=} \PY{n}{px}\PY{o}{.}\PY{n}{choropleth\PYZus{}mapbox}\PY{p}{(}\PY{n}{df\PYZus{}filtered}\PY{p}{,} \PY{n}{geojson}\PY{o}{=}\PY{n}{counties}\PY{p}{,} \PY{n}{locations}\PY{o}{=}\PY{l+s+s1}{\PYZsq{}}\PY{l+s+s1}{FIPS}\PY{l+s+s1}{\PYZsq{}}\PY{p}{,} \PY{n}{color}\PY{o}{=}\PY{l+s+s1}{\PYZsq{}}\PY{l+s+s1}{TOT\PYZus{}POP}\PY{l+s+s1}{\PYZsq{}}\PY{p}{,}  \PY{c+c1}{\PYZsh{} Updated column name}
                           \PY{n}{color\PYZus{}continuous\PYZus{}scale}\PY{o}{=}\PY{l+s+s2}{\PYZdq{}}\PY{l+s+s2}{electric}\PY{l+s+s2}{\PYZdq{}}\PY{p}{,}
                           \PY{n}{range\PYZus{}color}\PY{o}{=}\PY{p}{(}\PY{l+m+mi}{5000}\PY{p}{,} \PY{l+m+mi}{50000}\PY{p}{)}\PY{p}{,}
                           \PY{n}{mapbox\PYZus{}style}\PY{o}{=}\PY{l+s+s2}{\PYZdq{}}\PY{l+s+s2}{carto\PYZhy{}positron}\PY{l+s+s2}{\PYZdq{}}\PY{p}{,}
                           \PY{n}{zoom}\PY{o}{=}\PY{l+m+mi}{3}\PY{p}{,} \PY{n}{center} \PY{o}{=} \PY{p}{\PYZob{}}\PY{l+s+s2}{\PYZdq{}}\PY{l+s+s2}{lat}\PY{l+s+s2}{\PYZdq{}}\PY{p}{:} \PY{l+m+mf}{36.7782}\PY{p}{,} \PY{l+s+s2}{\PYZdq{}}\PY{l+s+s2}{lon}\PY{l+s+s2}{\PYZdq{}}\PY{p}{:} \PY{o}{\PYZhy{}}\PY{l+m+mf}{90.4179}\PY{p}{\PYZcb{}}\PY{p}{,}
                           \PY{n}{opacity}\PY{o}{=}\PY{l+m+mf}{0.5}\PY{p}{,}
                           \PY{n}{labels}\PY{o}{=}\PY{p}{\PYZob{}}\PY{l+s+s1}{\PYZsq{}}\PY{l+s+s1}{TOT\PYZus{}POP}\PY{l+s+s1}{\PYZsq{}}\PY{p}{:} \PY{l+s+s1}{\PYZsq{}}\PY{l+s+s1}{Total Population}\PY{l+s+s1}{\PYZsq{}}\PY{p}{\PYZcb{}}  \PY{c+c1}{\PYZsh{} Updated label name}
                          \PY{p}{)}
\PY{n}{fig}\PY{o}{.}\PY{n}{update\PYZus{}layout}\PY{p}{(}\PY{n}{margin}\PY{o}{=}\PY{p}{\PYZob{}}\PY{l+s+s2}{\PYZdq{}}\PY{l+s+s2}{r}\PY{l+s+s2}{\PYZdq{}}\PY{p}{:}\PY{l+m+mi}{0}\PY{p}{,}\PY{l+s+s2}{\PYZdq{}}\PY{l+s+s2}{t}\PY{l+s+s2}{\PYZdq{}}\PY{p}{:}\PY{l+m+mi}{0}\PY{p}{,}\PY{l+s+s2}{\PYZdq{}}\PY{l+s+s2}{l}\PY{l+s+s2}{\PYZdq{}}\PY{p}{:}\PY{l+m+mi}{0}\PY{p}{,}\PY{l+s+s2}{\PYZdq{}}\PY{l+s+s2}{b}\PY{l+s+s2}{\PYZdq{}}\PY{p}{:}\PY{l+m+mi}{0}\PY{p}{\PYZcb{}}\PY{p}{)}
\PY{n}{fig}\PY{o}{.}\PY{n}{show}\PY{p}{(}\PY{p}{)}
\end{Verbatim}
\end{tcolorbox}

    \begin{center}
    \adjustimage{max size={0.9\linewidth}{0.9\paperheight}}{output_10_0.png}
    \end{center}
    { \hspace*{\fill} \\}
    
    Q3. The Excel file ``ConSpend'' shows the consumer spending patterns
(Sales) by several variables including gender, when the purchase was
made (day), payment method type. Import the file. We are interesteed to
see if there is a difference between weekend and weekday spenings.
Create a new categorial variable ``Weekend'' that classifies the ``day''
into weekend if its Friday, Saturday or Sunday, and weekday otherwise.
(3 points)

    Using Seaborn, create a histogram showing Sales for weekend and
weekdays, with a transparanecy of 0.3 (2 points)

    \begin{tcolorbox}[breakable, size=fbox, boxrule=1pt, pad at break*=1mm,colback=cellbackground, colframe=cellborder]
\prompt{In}{incolor}{53}{\boxspacing}
\begin{Verbatim}[commandchars=\\\{\}]
\PY{k+kn}{import} \PY{n+nn}{seaborn} \PY{k}{as} \PY{n+nn}{sns}
\PY{n}{plt}\PY{o}{.}\PY{n}{style}\PY{o}{.}\PY{n}{use}\PY{p}{(}\PY{l+s+s1}{\PYZsq{}}\PY{l+s+s1}{default}\PY{l+s+s1}{\PYZsq{}}\PY{p}{)}
\PY{n}{conspend} \PY{o}{=} \PY{n}{pd}\PY{o}{.}\PY{n}{read\PYZus{}excel}\PY{p}{(}\PY{l+s+s1}{\PYZsq{}}\PY{l+s+s1}{ConSpend.xlsx}\PY{l+s+s1}{\PYZsq{}}\PY{p}{)}
\PY{n}{conspend}\PY{p}{[}\PY{l+s+s1}{\PYZsq{}}\PY{l+s+s1}{Weekend}\PY{l+s+s1}{\PYZsq{}}\PY{p}{]} \PY{o}{=} \PY{p}{[}\PY{l+s+s1}{\PYZsq{}}\PY{l+s+s1}{Weekend}\PY{l+s+s1}{\PYZsq{}} \PY{k}{if} \PY{n}{x} \PY{o+ow}{in} \PY{p}{[}\PY{l+s+s1}{\PYZsq{}}\PY{l+s+s1}{Sunday}\PY{l+s+s1}{\PYZsq{}}\PY{p}{,} \PY{l+s+s1}{\PYZsq{}}\PY{l+s+s1}{Saturday}\PY{l+s+s1}{\PYZsq{}}\PY{p}{,} \PY{l+s+s1}{\PYZsq{}}\PY{l+s+s1}{Friday}\PY{l+s+s1}{\PYZsq{}}\PY{p}{]} \PY{k}{else} \PY{l+s+s1}{\PYZsq{}}\PY{l+s+s1}{Weekday}\PY{l+s+s1}{\PYZsq{}} \PY{k}{for} \PY{n}{x} \PY{o+ow}{in} \PY{n}{conspend}\PY{p}{[}\PY{l+s+s1}{\PYZsq{}}\PY{l+s+s1}{Day}\PY{l+s+s1}{\PYZsq{}}\PY{p}{]}\PY{p}{]}
\PY{n}{sns}\PY{o}{.}\PY{n}{histplot}\PY{p}{(}\PY{n}{conspend}\PY{p}{,} \PY{n}{x}\PY{o}{=}\PY{l+s+s2}{\PYZdq{}}\PY{l+s+s2}{Sales}\PY{l+s+s2}{\PYZdq{}}\PY{p}{,} \PY{n}{hue}\PY{o}{=}\PY{l+s+s2}{\PYZdq{}}\PY{l+s+s2}{Weekend}\PY{l+s+s2}{\PYZdq{}}\PY{p}{,} \PY{n}{element}\PY{o}{=}\PY{l+s+s2}{\PYZdq{}}\PY{l+s+s2}{step}\PY{l+s+s2}{\PYZdq{}}\PY{p}{,} \PY{n}{alpha} \PY{o}{=} \PY{l+m+mf}{0.3}\PY{p}{)}
\end{Verbatim}
\end{tcolorbox}

            \begin{tcolorbox}[breakable, size=fbox, boxrule=.5pt, pad at break*=1mm, opacityfill=0]
\prompt{Out}{outcolor}{53}{\boxspacing}
\begin{Verbatim}[commandchars=\\\{\}]
<Axes: xlabel='Sales', ylabel='Count'>
\end{Verbatim}
\end{tcolorbox}
        
    \begin{center}
    \adjustimage{max size={0.9\linewidth}{0.9\paperheight}}{output_13_1.png}
    \end{center}
    { \hspace*{\fill} \\}
    
    \begin{figure}
\centering
\includegraphics{attachment:Hist1.png}
\caption{Hist1.png}
\end{figure}

    Q4. Using plotly, create overlapping histograms that show the sales by
weekdays and weekends, with an opacity of 0,3. Pass the `marginal'
argument into the function to also show a ``box'' plot.'' Also, set
x-axis ticks to `50' and the plot background color to white (4 points).

    \begin{tcolorbox}[breakable, size=fbox, boxrule=1pt, pad at break*=1mm,colback=cellbackground, colframe=cellborder]
\prompt{In}{incolor}{114}{\boxspacing}
\begin{Verbatim}[commandchars=\\\{\}]
\PY{k+kn}{import} \PY{n+nn}{plotly}\PY{n+nn}{.}\PY{n+nn}{express} \PY{k}{as} \PY{n+nn}{px}
\PY{n}{fig} \PY{o}{=} \PY{n}{px}\PY{o}{.}\PY{n}{histogram}\PY{p}{(}\PY{n}{conspend}\PY{p}{,} \PY{n}{x}\PY{o}{=}\PY{l+s+s2}{\PYZdq{}}\PY{l+s+s2}{Sales}\PY{l+s+s2}{\PYZdq{}}\PY{p}{,} \PY{n}{color}\PY{o}{=}\PY{l+s+s2}{\PYZdq{}}\PY{l+s+s2}{Weekend}\PY{l+s+s2}{\PYZdq{}}\PY{p}{,}
                   \PY{n}{marginal}\PY{o}{=}\PY{l+s+s2}{\PYZdq{}}\PY{l+s+s2}{box}\PY{l+s+s2}{\PYZdq{}}\PY{p}{,} \PY{c+c1}{\PYZsh{} or violin, rug}
                   \PY{n}{hover\PYZus{}data}\PY{o}{=}\PY{n}{conspend}\PY{o}{.}\PY{n}{columns}\PY{p}{,}
                   \PY{n}{width}\PY{o}{=}\PY{l+m+mi}{800}\PY{p}{,} \PY{n}{height}\PY{o}{=}\PY{l+m+mi}{600}\PY{p}{,}
                   \PY{n}{title}\PY{o}{=}\PY{l+s+s1}{\PYZsq{}}\PY{l+s+s1}{Sales by Weekend versus Weekdays}\PY{l+s+s1}{\PYZsq{}}\PY{p}{,}
                   \PY{c+c1}{\PYZsh{} template=\PYZsq{}simple\PYZus{}white\PYZsq{}, }
                   \PY{n}{opacity}\PY{o}{=}\PY{l+m+mf}{.3}
                   \PY{p}{)}
\PY{n}{fig}\PY{o}{.}\PY{n}{show}\PY{p}{(}\PY{p}{)}
\end{Verbatim}
\end{tcolorbox}

    \begin{center}
    \adjustimage{max size={0.9\linewidth}{0.9\paperheight}}{output_16_0.png}
    \end{center}
    { \hspace*{\fill} \\}
    
    \begin{figure}
\centering
\includegraphics{attachment:newplot\%284\%29.png}
\caption{newplot\%284\%29.png}
\end{figure}

    Q5. Using Seaborn, create boxplots showing Sales for weekend and
weekdays and eliminated the outliers (2 points)

    \begin{tcolorbox}[breakable, size=fbox, boxrule=1pt, pad at break*=1mm,colback=cellbackground, colframe=cellborder]
\prompt{In}{incolor}{87}{\boxspacing}
\begin{Verbatim}[commandchars=\\\{\}]
\PY{k+kn}{import} \PY{n+nn}{seaborn} \PY{k}{as} \PY{n+nn}{sns}
\PY{n}{plt}\PY{o}{.}\PY{n}{figure}\PY{p}{(}\PY{n}{figsize}\PY{o}{=}\PY{p}{(}\PY{l+m+mi}{5}\PY{p}{,} \PY{l+m+mi}{5}\PY{p}{)}\PY{p}{)}
\PY{n}{sns}\PY{o}{.}\PY{n}{boxplot}\PY{p}{(}\PY{n}{conspend}\PY{p}{,} \PY{n}{y} \PY{o}{=} \PY{l+s+s1}{\PYZsq{}}\PY{l+s+s1}{Sales}\PY{l+s+s1}{\PYZsq{}}\PY{p}{,} \PY{n}{x} \PY{o}{=} \PY{l+s+s1}{\PYZsq{}}\PY{l+s+s1}{Weekend}\PY{l+s+s1}{\PYZsq{}}\PY{p}{,}\PY{n}{showfliers}\PY{o}{=}\PY{k+kc}{False}\PY{p}{)}
\PY{n}{sns}\PY{o}{.}\PY{n}{despine}\PY{p}{(}\PY{p}{)}
\end{Verbatim}
\end{tcolorbox}

    \begin{center}
    \adjustimage{max size={0.9\linewidth}{0.9\paperheight}}{output_19_0.png}
    \end{center}
    { \hspace*{\fill} \\}
    
    \begin{figure}
\centering
\includegraphics{attachment:Bar1.png}
\caption{Bar1.png}
\end{figure}

    Q6. Using plotly, create box plots to show weekend versu weekday sales,
differentiated by color.Alos show the distribution of data points on the
plot (2 points).

    \begin{tcolorbox}[breakable, size=fbox, boxrule=1pt, pad at break*=1mm,colback=cellbackground, colframe=cellborder]
\prompt{In}{incolor}{94}{\boxspacing}
\begin{Verbatim}[commandchars=\\\{\}]
\PY{k+kn}{import} \PY{n+nn}{plotly}\PY{n+nn}{.}\PY{n+nn}{express} \PY{k}{as} \PY{n+nn}{px}
\PY{n}{fig} \PY{o}{=} \PY{n}{px}\PY{o}{.}\PY{n}{box}\PY{p}{(}\PY{n}{conspend}\PY{p}{,} \PY{n}{x}\PY{o}{=}\PY{l+s+s2}{\PYZdq{}}\PY{l+s+s2}{Weekend}\PY{l+s+s2}{\PYZdq{}}\PY{p}{,} \PY{n}{y}\PY{o}{=}\PY{l+s+s2}{\PYZdq{}}\PY{l+s+s2}{Sales}\PY{l+s+s2}{\PYZdq{}}\PY{p}{,} \PY{n}{points}\PY{o}{=}\PY{l+s+s1}{\PYZsq{}}\PY{l+s+s1}{all}\PY{l+s+s1}{\PYZsq{}}\PY{p}{,} \PY{n}{color}\PY{o}{=}\PY{l+s+s2}{\PYZdq{}}\PY{l+s+s2}{Weekend}\PY{l+s+s2}{\PYZdq{}}\PY{p}{,} \PY{n}{color\PYZus{}discrete\PYZus{}sequence}\PY{o}{=}\PY{p}{[}\PY{l+s+s1}{\PYZsq{}}\PY{l+s+s1}{blue}\PY{l+s+s1}{\PYZsq{}}\PY{p}{,} \PY{l+s+s1}{\PYZsq{}}\PY{l+s+s1}{green}\PY{l+s+s1}{\PYZsq{}}\PY{p}{]}\PY{p}{,} \PY{n}{width}\PY{o}{=}\PY{l+m+mi}{1000}\PY{p}{,} \PY{n}{height}\PY{o}{=}\PY{l+m+mi}{600}\PY{p}{)}
\PY{n}{fig}\PY{o}{.}\PY{n}{show}\PY{p}{(}\PY{p}{)}
\end{Verbatim}
\end{tcolorbox}

    \begin{center}
    \adjustimage{max size={0.9\linewidth}{0.9\paperheight}}{output_22_0.png}
    \end{center}
    { \hspace*{\fill} \\}
    
    \begin{figure}
\centering
\includegraphics{attachment:newplot\%285\%29.png}
\caption{newplot\%285\%29.png}
\end{figure}

    Q7. Generate the summary statistics for weekend and weekday sales. Is
there a difference in the value of the sample means? (2 points)

    \begin{tcolorbox}[breakable, size=fbox, boxrule=1pt, pad at break*=1mm,colback=cellbackground, colframe=cellborder]
\prompt{In}{incolor}{95}{\boxspacing}
\begin{Verbatim}[commandchars=\\\{\}]
\PY{n}{conspend}\PY{p}{[}\PY{p}{[}\PY{l+s+s1}{\PYZsq{}}\PY{l+s+s1}{Weekend}\PY{l+s+s1}{\PYZsq{}}\PY{p}{,} \PY{l+s+s1}{\PYZsq{}}\PY{l+s+s1}{Sales}\PY{l+s+s1}{\PYZsq{}}\PY{p}{]}\PY{p}{]}\PY{o}{.}\PY{n}{groupby}\PY{p}{(}\PY{l+s+s1}{\PYZsq{}}\PY{l+s+s1}{Weekend}\PY{l+s+s1}{\PYZsq{}}\PY{p}{)}\PY{o}{.}\PY{n}{describe}\PY{p}{(}\PY{p}{)}
\end{Verbatim}
\end{tcolorbox}

            \begin{tcolorbox}[breakable, size=fbox, boxrule=.5pt, pad at break*=1mm, opacityfill=0]
\prompt{Out}{outcolor}{95}{\boxspacing}
\begin{Verbatim}[commandchars=\\\{\}]
         Sales
         count        mean         std    min    25\%     50\%     75\%     max
Weekend
Weekday  217.0  148.889954   94.933225  14.76  83.15  124.21  190.74  463.40
Weekend  183.0  159.661694  100.135681   6.82  84.72  132.06  219.73  485.01
\end{Verbatim}
\end{tcolorbox}
        
    Summary: Seeing the summary statistics, there is difference of sample
means value between weekend and weekday

    Q8. At an alpha of 0.05, is there a statitsical difference between sales
on weekday versus weekends? Show your statitscal test and explain. (4
points)

    \begin{tcolorbox}[breakable, size=fbox, boxrule=1pt, pad at break*=1mm,colback=cellbackground, colframe=cellborder]
\prompt{In}{incolor}{99}{\boxspacing}
\begin{Verbatim}[commandchars=\\\{\}]
\PY{n}{stats}\PY{o}{.}\PY{n}{ttest\PYZus{}ind}\PY{p}{(}\PY{n}{a}\PY{o}{=}\PY{n}{conspend}\PY{p}{[}\PY{n}{conspend}\PY{o}{.}\PY{n}{Weekend} \PY{o}{==} \PY{l+s+s1}{\PYZsq{}}\PY{l+s+s1}{Weekend}\PY{l+s+s1}{\PYZsq{}}\PY{p}{]}\PY{p}{[}\PY{l+s+s1}{\PYZsq{}}\PY{l+s+s1}{Sales}\PY{l+s+s1}{\PYZsq{}}\PY{p}{]}\PY{p}{,} \PY{n}{b}\PY{o}{=}\PY{n}{conspend}\PY{p}{[}\PY{n}{conspend}\PY{o}{.}\PY{n}{Weekend} \PY{o}{==} \PY{l+s+s1}{\PYZsq{}}\PY{l+s+s1}{Weekday}\PY{l+s+s1}{\PYZsq{}}\PY{p}{]}\PY{p}{[}\PY{l+s+s1}{\PYZsq{}}\PY{l+s+s1}{Sales}\PY{l+s+s1}{\PYZsq{}}\PY{p}{]}\PY{p}{,} \PY{n}{equal\PYZus{}var}\PY{o}{=}\PY{k+kc}{True}\PY{p}{)}
\end{Verbatim}
\end{tcolorbox}

            \begin{tcolorbox}[breakable, size=fbox, boxrule=.5pt, pad at break*=1mm, opacityfill=0]
\prompt{Out}{outcolor}{99}{\boxspacing}
\begin{Verbatim}[commandchars=\\\{\}]
TtestResult(statistic=1.1025285650269985, pvalue=0.2708981806010835, df=398.0)
\end{Verbatim}
\end{tcolorbox}
        
    Summary: Seeing the statistics test, there is no statistical difference
of sample means value between weekend and weekday since the p-value
obtained from the test is above 0.05


    % Add a bibliography block to the postdoc
    
    
    
\end{document}
